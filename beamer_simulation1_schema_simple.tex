\documentclass[aspectratio=169]{beamer}
\usetheme{Madrid}
\usecolortheme{default}

\usepackage[T1]{fontenc}
\usepackage[utf8]{inputenc}
\usepackage[french]{babel}

\usepackage{tikz}
\usetikzlibrary{arrows.meta, positioning, shapes.geometric}
\usepackage{adjustbox}

\title{Schémas simplifiés — Simulations Monte Carlo}
\author{}
\date{}

\tikzset{
  arrow/.style={-Latex, very thick},
  box/.style={rectangle, rounded corners, draw, align=center, inner sep=12pt},
  in/.style={box, fill=blue!6, text width=0.28\textwidth},
  mid/.style={box, fill=orange!12, text width=0.34\textwidth},
  out/.style={box, fill=green!10, text width=0.28\textwidth}
}

\begin{document}

\begin{frame}
  \titlepage
\end{frame}

\begin{frame}[plain]{Simulation 1 — Risque de rupture de stock (très simple)}
\vspace{-2mm}
\centering
\begin{adjustbox}{max totalsize={\textwidth}{0.62\textheight},center}
\begin{tikzpicture}[font=\Large, node distance=16mm]

\node[in] (inp) {\textbf{Entrées}\par\medskip
Stocks\par
Délais\par
Règles de commande\par
Plan de production};

\node[mid, right=of inp, xshift=8mm] (sys) {\textbf{Simulation}\par\medskip
On rejoue \textbf{N fois} la réalité\par
(mois par mois)\par
avec des aléas};

\node[out, right=of sys, xshift=8mm] (out) {\textbf{Résultat}\par\medskip
Pour chaque matière :\par
\textbf{risque de rupture}\par
(\% de cas)};

\draw[arrow] (inp) -- (sys);
\draw[arrow] (sys) -- (out);

\end{tikzpicture}
\end{adjustbox}
\end{frame}

\begin{frame}[plain]{Simulation 2 — Délai pour atteindre 11\,000--15\,000 t/an (très simple)}
\vspace{-2mm}
\centering
\begin{adjustbox}{max totalsize={\textwidth}{0.62\textheight},center}
\begin{tikzpicture}[font=\Large, node distance=16mm]

\node[in] (inp) {\textbf{Entrées}\par\medskip
Marché\par
Capacité\par
Clients / demande\par
Cible 11--15 kt/an};

\node[mid, right=of inp, xshift=8mm] (sys) {\textbf{Simulation}\par\medskip
On rejoue \textbf{N fois}\par
le ramp-up (mois par mois)\par
avec des aléas};

\node[out, right=of sys, xshift=8mm] (out) {\textbf{Résultat}\par\medskip
Quand on atteint\par
11 kt/an et 15 kt/an\par
(délai probable)};

\draw[arrow] (inp) -- (sys);
\draw[arrow] (sys) -- (out);

\end{tikzpicture}
\end{adjustbox}
\end{frame}

\end{document}
